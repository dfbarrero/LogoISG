\documentclass[border=0pt]{standalone}
\usepackage[utf8]{inputenc}
\usepackage{fix-cm}
\usepackage{microtype}
\usepackage{tikz}
\usetikzlibrary{calc, arrows, intersections}

\title{Logo ISG}

%%%% DESCOMENTAR ESQUEMA DE COLORES DESEADO %%%%%
% Fondo a color
\definecolor{fondo}{RGB}{35,43,67}
\definecolor{fuente}{RGB}{255,255,255}

\definecolor{fuente}{RGB}{35,43,67}
\definecolor{fondo}{RGB}{255,255,255}


% Blanco y negro
%\definecolor{fondo}{RGB}{255,255,255}
%\definecolor{fuente}{RGB}{0,0,0}

% Negativo
%\definecolor{fondo}{RGB}{0,0,0}
%\definecolor{fuente}{RGB}{255,255,255}
%%%%%%%%%%
\begin{document}

\tikzstyle{circulo}=[draw, circle, inner sep=0, minimum width=1cm] % minimum width contiene el diametro del nodo, debe ser el doble que la orden arc

\begin{tikzpicture}[line width=4.5pt, cap=round, fuente]
%\draw[help lines] (-3,-3) grid (10,3);

\draw[fill=fondo, fondo] (-3,-3) rectangle (10,3);

\path 	(0,0) coordinate (centro)
		(90:2) coordinate (arriba) 
		(-30:2) coordinate (derecha) 
		(210:2) coordinate (izquierda);

% Dibujamos círculos
\draw[name path=arcocentro] (45:1) arc[radius=1cm, start angle=45, end angle= 385];
\node[circulo, name path=arcoarriba] (arribanodo) at (arriba) {};
\draw[name path=arcoizquierda] (izquierda) ++(-45:0.5) arc[radius=0.5cm, start angle=-45, end angle= 280];
\node[circulo, name path=arcoderecha] (derechanodo) at (derecha) {};

% Unimos centro de círculos
\path[name path=lineaarriba] (arribanodo) -- (centro);
\fill [name intersections={of= arcocentro and lineaarriba}] (intersection-1) coordinate (puntoA);
\draw (arribanodo.south) -- (puntoA);

\path[name path=lineaizquierda] (centro) -- (izquierda);
\fill [name intersections={of= arcoizquierda and lineaizquierda}] (intersection-1) coordinate (puntoI);
\path (centro) -- (puntoI) coordinate[pos=0.45] (cutI);
\draw (puntoI) -- (cutI);

\path[thin, name path=lineaderecha] (centro) -- (derecha);
\fill [name intersections={of= arcoderecha and lineaderecha}] (intersection-1) coordinate (puntoD);
\path (centro) -- (puntoD) coordinate[pos=0.45] (cutD);
\draw (puntoD) -- (cutD);

% Líneas de ayuda
%\fill [green] (puntoD) circle (2pt);
%\fill [green] (puntoI) circle (2pt);
%\fill [green] (puntoA) circle (2pt);

%\fill [red] (arriba) circle (2pt);
%\fill [red] (izquierda) circle (2pt);
%\fill [red] (derecha) circle (2pt);
%\fill [red] (centro) circle (2pt);
%\draw[thin] (0,0) -- (45:2);
%\draw[thin] (0,0) -- (30:2);
%\draw[thin] (izquierda) -- ++(-45:2);
%\draw[thin] (izquierda) -- ++(-70:2);

% Texto
\node at (5.5,0.25) (isg) {\fontsize{100}{200}\selectfont \textls[100]{ISG}};
\path (derechanodo.south -| isg) coordinate (A);
\draw[thin] (A) ++(-1.5,0) -- ++(3,0);
\node[below of = A, yshift=0.3cm] (texto) {\fontsize{18}{80}\selectfont \textls[100]{Intelligent Systems Group}};

\end{tikzpicture}
\end{document}
